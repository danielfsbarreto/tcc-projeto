\documentclass[a4paper,11pt]{article}
\usepackage[left=3.0cm,right=2.5cm,top=3.0cm]{geometry}
\setlength{\parindent}{.0in}
\parskip 10pt        
\usepackage[utf8]{inputenc}
\usepackage[brazil]{babel}
\usepackage{amssymb}
\usepackage{amsthm}
\usepackage{hyperref}
\usepackage{color}
\usepackage{graphicx}
\usepackage{multirow}
\usepackage{hhline}
\usepackage{float}
\usepackage[table]{xcolor}
        
\newcommand{\D}{\displaystyle}
\newcommand{\mc}{\multicolumn}
\newcommand{\ft}{\footnotesize}
\newcommand{\Sc}{\scriptsize}
\newcommand{\rset}{\mathbb{R}}
\newcommand{\zset}{\mathbb{Z}}

\newcommand\SB[1]{{\color{cyan} #1}}
\newcommand\JC[1]{{\color{red} #1}}
\newcommand\AO[1]{{\color{green} #1}}

\newcounter{contadorDeEtapas}

\begin{document}
\pagestyle {empty}

\vspace*{-2cm}
\begin{figure}[h]
\leavevmode
\begin{minipage}[t]{\textwidth}
\includegraphics[scale=0.7]{logo-ufrpe.eps}
\end{minipage}
\end{figure}

\vspace*{-3.0cm}
{\bf
\begin{center}
{
\hspace*{0cm} 	MINISTÉRIO DA EDUCAÇÃO E DO DESPORTO \\
\hspace*{.2in} UNIVERSIDADE FEDERAL RURAL DE PERNAMBUCO \\
\hspace*{.2in} BACHARELADO EM SISTEMAS DE INFORMAÇÃO} \\
\end{center}}
\vspace{0.0cm}
\noindent
\begin{figure}[h]
\centering
\includegraphics[scale=0.5]{Logo-bsi-presencial-v3-amp.eps}
\end{figure}
\vspace*{2.0cm}
\begin{center}

{\Large \bf  Projeto de Conclusão de Curso}\\[1cm]
{\Large \bf Título: Um Estudo sobre Desafios e Benefícios na Adoção de Métodos Ágeis por Organizações de Software} \\[3cm]
\end{center}
{\Large  Estudante: Daniel Filype Silva Barreto}\\[6mm]
{\Large  Orientador: Teresa Maciel}\\[6mm]

\vspace{3.0cm}
\begin{center}
{\large {\bf Recife}\\[6mm]
Janeiro de 2014}
\end{center}
\newpage

\tableofcontents

\newpage
\pagestyle {plain}
\setcounter{page}{0} \pagenumbering{arabic}

\section{Introdução}
\subsection{Apresentação}
O Manifesto Ágil \cite{agileManifesto}, criado em 2001, revolucionou o mercado de Tecnologia da Informação, pois, até então, a maneira de se desenvolver software mostrou-se falha em diversos aspectos. Contudo, a transição entre o modelo antigo (conhecido como ``cascata") e o moderno (conhecido como ``ágil") é uma tarefa não-trivial.

As principais motivações por trás da adoção ágil de desenvolvimento de software são: a melhoria da qualidade do produto final, aumento da moral dos desenvolvedores e satisfação do cliente. Entretanto, adoção ágil sempre vem com desafios especiais e, para que ela ocorra com sucesso, mudanças fundamentais na cultura da organização são necessárias \cite{Hassan2011}. Este projeto visa expor desafios e lições aprendidas por empresas brasileiras de desenvolvimento de software que passaram por este processo.
\subsection{Justificativas}
Muitas abordagens diferentes podem ser aplicadas a desenvolvimento de software \cite{Kettunen2010}, cada uma com suas peculiaridades. De acordo com \cite{Shore2007}, metodologias denominadas ágeis tornaram-se bastante populares, sendo utilizadas por grandes corporações como: Google, Yahoo!, Symantec, Microsoft, etc. Entretanto, não existe bala de prata no que tange a desenvolvimento de software. Não se deve simplesmente aplicá-las porque são populares. Existem vários estudos na literatura que mostram casos falhos de adoção de métodos ágeis \cite{Krasteva2008}. É preciso entender a fundo estas novas metodologias, analisar seus pontos positivos e negativos e, principalmente, refletir se elas agregariam valor ao produto final a ser entregue.

Sendo assim, é de suma importância a coleta de experiências para que se possa aprender sobre esta transição com aqueles que já passaram por este processo e que, além disso, querem contribuir com a comunidade de desenvolvimento de software.
\subsection{Objetivos}
\subsubsection{Objetivo Geral}
\begin{itemize}
	\item Disponibilizar um conjunto de benefícios e desafios mais relatados na adoção do desenvolvimento ágil por organizações de software.
\end{itemize}
\subsubsection{Objetivos Específicos}
\begin{itemize}
	\item Investigar, através de pesquisa exploratória, a necessidade e contribuição de trabalhos desta natureza.
	\item Adquirir conhecimento sobre fundamentos teóricos relacionados como desenvolvimento ágil de software e lean.
	\item Realizar uma revisão utilizando como base a metodologia de Revisão Sistemática \cite{Barbara2004} para coletar lições aprendidas provenientes de trabalhos publicados pela comunidade científica.
	\item Realizar uma pesquisa exploratória para coletar lições aprendidas provenientes de trabalhos publicados pela indústria de software nas principais conferências neste contexto.
	\item Identificar um conjunto de benefícios e desafios mais citados a partir dos resultados das pesquisas realizadas.
	\item Validar o trabalho realizado através de metodologias de survey, aplicando-se questionários e entrevistas com especialistas nacionais reconhecidos na área, assim como em empresas que adotam o desenvolvimento ágil.
	\item Gerar um protótipo de um banco de dados de lições aprendidas para acesso livre pela comunidade de software.
\end{itemize}
\subsection{Organização do Trabalho}
Este trabalho é composto por mais quatro seções. Na Seção 2 está exposto uma introdução sobre metodologias ágeis. Na seção 3,  encontra-se o detalhamento da metodologia usada para realização do trabalho. O cronograma do projeto está exposto na sessão 4. Finalizando, encontram-se as referências bibliográficas de todo o material consultado para a realização deste documento.

\section{Desenvolvimento de Software e Adoção Ágil}
Como já foi mencionado anteriormente, ocorreu uma reviravolta no mundo do desenvolvimento de software em 2001. Era o Manifesto Ágil \cite{agileManifesto}. Seus criadores estavam descobrindo maneiras melhores de desenvolver software de forma colaborativa, seguindo os seguintes valores:
\begin{itemize}
	\item Indivíduos e interações mais que processos e ferramentas
	\item Software em funcionamento mais que documentação abrangente
	\item Colaboração com o cliente mais que negociação de contratos
	\item Responder a mudanças mais que seguir um plano
\end{itemize}

Além destes quatro princípios, foram idealizados os Doze Princípios do Software Ágil. São eles:
\begin{enumerate}
	\item Nossa maior prioridade é satisfazer o cliente, através da entrega adiantada e contínua de software de valor.
	\item Aceitar mudanças de requisitos, mesmo no fim do desenvolvimento. Processos ágeis se adequam a mudanças, para que o cliente possa tirar vantagens competitivas.
	\item Entregar software funcionando com frequência, na escala de semanas até meses, com preferência aos períodos mais curtos.
	\item Pessoas relacionadas à negócios e desenvolvedores devem trabalhar em conjunto e diariamente, durante todo o curso do projeto.
	\item Construir projetos ao redor de indivíduos motivados. Dando a eles o ambiente e suporte necessário, e confiar que farão seu trabalho.
	\item O Método mais eficiente e eficaz de transmitir informações para, e por dentro de um time de desenvolvimento, é através de uma conversa cara a cara.
	\item Software funcional é a medida primária de progresso.
	\item Processos ágeis promovem um ambiente sustentável. Os patrocinadores, desenvolvedores e usuários, devem ser capazes de manter indefinidamente, passos constantes.
	\item Contínua atenção à excelência técnica e bom design, aumenta a agilidade.
	\item Simplicidade: a arte de maximizar a quantidade de trabalho que não precisou ser feito.
	\item As melhores arquiteturas, requisitos e designs emergem de times auto-organizáveis.
	\item Em intervalos regulares, o time reflete em como ficar mais efetivo, então, se ajustam e otimizam seu comportamento de acordo.
\end{enumerate}

Percebe-se pelos princípios e valores citados no manifesto que ágil não é apenas uma maneira de desenvolver software, e sim uma filosofia. Passar a ser ágil não é fácil, é uma mudança na maneira como a organização trabalha, uma mudança de mentalidade. Para se tirar proveito dos benefícios de uma transformação ágil, toda a organização precisa apoiar e participar ativamente da transição \cite{Kirsi2013}. Mas os desafios não param por aí. Problemas como falta de experiência, comunicação entre times distribuídos, disponibilidade do PO, escalabilidade, falta de abertura a mudanças e  outras dificuldades de quebra de paradigma são situações bastante comuns.

Adotar ágil é comumente um processo longo e que não existe fórmula para o sucesso \cite{Block2011}. Entretanto, muitos relatos publicados na literatura por empresas bastante conhecidas \cite{Adobe2012,Microsoft2013,NHN2012} podem servir como referência para empresas que têm interesse em aderir ao movimento.

\section{Metodologia}
O projeto será realizado seguindo uma metodologia dividida em diversas etapas:
\begin{list}{\bfseries{}E\arabic{contadorDeEtapas} -~}{\usecounter{contadorDeEtapas}\bfseries}
		\item Estudo exploratório sobre desenvolvimento ágil para identificação do objetivo do trabalho:

			\textnormal{Fase cujo foco é identificar pontos em aberto na literatura para se trabalhar dentro do contexto de \textit{Agile Adoption}.}

		\item Elaboração das questões de pesquisa:

			\textnormal{A partir do conhecimento adquirido no estudo realizado no item anterior, serão elaborados os questionamentos que serão levantados pelas pesquisas subsequentes e validados por profissionais atuantes no mercado brasileiro de Tecnologia da Informação.}

		\item Fundamentação teórica (métodos ágeis, lean, adoção ágil, etc.):

			\textnormal{Fase de estudo exploratório sobre metodologias ágeis, lean e o contexto envolvido na transição de uma empresa de software para adoção dos mesmos. Serão analisados aspectos relevantes do contexto a ser explorado relatados em livros e conferências para um melhor entendimento do assunto.}

		\item Investigação de desafios e lições aprendidas  a partir de publicações científicas:

			\textnormal{Utilizando um protocolo de pesquisa baseado na metodologia de revisão sistemática proposta por \cite{Barbara2004}, serão investigados artigos científicos publicados na literatura que relatem o processo de adoção ágil de empresas de desenvolvimento de software ao redor do mundo. Os desafios e lições aprendidas encontrados serão devidamente coletados e mapeados.}

		\item Investigação de desafios e lições aprendidas  a partir de relatos de experiência:

			\textnormal{Assim como na etapa anterior, relatos publicados nas principais conferências de engenharia de software serão analisados. Neste caso, será utilizada uma metodologia de revisão exploratória.}

		\item Consolidação dos resultados obtidos das revisões, análise de similaridades e categorização:

			\textnormal{Após ter sido feito o mapeamento dos dados coletados através da análise de publicações científicas e relatos de experiência, suas similaridades serão devidamente mapeadas e categorizadas através de uma análise qualitativa. O produto final desta etapa servirá como base para a elaboração do questionário da etapa seguinte.}

		\item Questionário e entrevistas com especialistas e empresas:

			\textnormal{O propósito desta etapa é validar o material fruto das análises feitas anteriormente. O questionário será elaborado seguindo uma metodologia de survey a ser definida.}

		\item Protótipo de um repositório de lições aprendidas:

			\textnormal{Etapa em que será criado um protótipo de um website cujo propósito é compilar os principais achados desta pesquisa e disponibilizá-lo em um único local. Todos da comunidade de desenvolvimento de software interessados pelo assunto poderão acessá-lo, tanto seu conteúdo como o código-fonte.}

		\item Análise geral do trabalho e identificação de trabalhos futuros:

			\textnormal{Fase de reflexão e análise para geração de possíveis trabalhos de pós-graduação.}

\end{list}

\section{Cronograma}

\begin{table}[H]
	\centering
	\begin{tabular}{| p{8cm} | c | c | c | c | c |} \hline
	\multirow{2}{*}{\textbf{Atividades}} & \multicolumn{2}{c |}{\textbf{2013}} & \multicolumn{3}{c |}{\textbf{2014}} \\ \cline{2-6}
		& \textbf{N} & \textbf{D} & \textbf{J} & \textbf{F} & \textbf{M} \\
		\hhline{------}
		E1. Estudo exploratório sobre desenvolvimento ágil para identificação do objetivo do trabalho & \cellcolor{blue!60} & & & & \\ \cline{1-1}
		E2. Elaboração das questões de pesquisa & \cellcolor{blue!60} & & & & \\ \cline{1-1}
		E3. Fundamentação teórica (métodos ágeis, lean, adoção ágil, etc.) & & \cellcolor{blue!60} & & & \\ \cline{1-1}
		E4. Investigação de desafios e lições aprendidas  a partir de publicações científicas & & & \cellcolor{blue!60} & & \\ \cline{1-1}
		E5. Investigação de desafios e lições aprendidas  a partir de relatos de experiência & & & \cellcolor{blue!60} & & \\ \cline{1-1}
		E6. Consolidação dos resultados obtidos das revisões, análise de similaridades e categorização & & & & \cellcolor{blue!60} & \\ \cline{1-1}
		E7. Questionário e entrevistas com especialistas e empresas & & & & \cellcolor{blue!60} & \\ \cline{1-1}
		E8. Protótipo de um repositório de lições aprendidas & & & & & \cellcolor{blue!60} \\ \cline{1-1}
		E9. Análise geral do trabalho e identificação de trabalhos futuros & & & & & \cellcolor{blue!60} \\ \hline
	\end{tabular}
\end{table}

\newpage
\addcontentsline{toc}{section}{Referências}
\bibliographystyle{plain}
\bibliography{referencias}

\end{document}
