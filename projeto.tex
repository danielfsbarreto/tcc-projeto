%=======================================================================
% Template para projeto de pesquisa
% Programa de Pós Graduação em Informática Aplicada da Universidade Federal 
% Rural de Pernambuco
%=======================================================================
\documentclass[a4paper,11pt]{article}
\usepackage[left=3.0cm,right=2.5cm,top=3.0cm]{geometry}
%espaçamento entre parágrafos
\setlength{\parindent}{.0in}
\parskip 10pt        
\usepackage[utf8]{inputenc}
\usepackage[brazil]{babel}
%% The amssymb package provides various useful mathematical symbols
\usepackage{amssymb}
%% The amsthm package provides extended theorem environments
\usepackage{amsthm}
\usepackage{hyperref}
\usepackage{color}
%\usepackage{backref}
\usepackage{graphicx}
        
\newcommand{\D}{\displaystyle}
\newcommand{\mc}{\multicolumn}
\newcommand{\ft}{\footnotesize}
\newcommand{\Sc}{\scriptsize}
\newcommand{\rset}{\mathbb{R}}
\newcommand{\zset}{\mathbb{Z}}

\newcommand\SB[1]{{\color{cyan} #1}}
\newcommand\JC[1]{{\color{red} #1}}
\newcommand\AO[1]{{\color{green} #1}}




\begin{document}
\pagestyle {empty}


%início da capa
\vspace*{-2cm}
\begin{figure}[h]
\leavevmode
\begin{minipage}[t]{\textwidth}
\includegraphics[scale=0.7]{logo-ufrpe.eps}
\end{minipage}
\end{figure}

\vspace*{-3.0cm}
{\bf
\begin{center}
{
\hspace*{0cm} 	MINISTÉRIO DA EDUCAÇÃO E DO DESPORTO \\
\hspace*{.2in} UNIVERSIDADE FEDERAL RURAL DE PERNAMBUCO \\
\hspace*{.2in} BACHARELADO EM SISTEMAS DE INFORMAÇÃO} \\
\end{center}}
\vspace{0.0cm}
\noindent
\begin{figure}[h]
\centering
\includegraphics[scale=0.5]{Logo-bsi-presencial-v3-amp.eps}
\end{figure}
\vspace*{2.0cm}
\begin{center}


{\Large \bf  Projeto de Conclusão de Curso}\\[1cm]
{\Large \bf Título:kjgktjgkt} \\[3cm]
\end{center}
{\Large  Estudante: Fulano de tal}\\[6mm]
{\Large  Orientador: }\\[6mm]



\vspace{3.0cm}
\begin{center}
{\large {\bf Recife}\\[6mm]
Novembro de 2013}
\end{center}
\newpage

\pagestyle {plain}
\setcounter{page}{0} \pagenumbering{arabic}

Para elaborar seu projeto de pesquisa consulte \cite{Silva05}. 


\section{Apresentação}
Definição e contextualização do problema a ser resolvido.
\section{Relevância}
Qual a importância ou relevância do tema?  Qual a contribuição para o meio acadêmico e para a sociedade do desenvolvimento desta pesquisa? Qual a vantagem em relação aos trabalhos relacionados?
\section{Objetivos}
\subsection{Objetivo Geral}
Indicar de forma genérica qual o objetivo principal a ser alcançado. Deve ser claro e conciso.
\subsection{bla}
Detalhar o objetivo geral mostrando o que se pretende alcançar. Tornar operacional o objetivo geral, indicando exatamente o que será realizado visando a atingir o mesmo.
Os objetivos devem iniciar sempre com um verbo no infinitivo: Proporcionar, Elaborar, Apresentar, Desenvolver, Aplicar, Conhecer, Identificar, etc.

\section{Referencial Teórico  (Estado da arte, revisão da literatura, revisão bibliográfica)} 
Deve-se realizar uma análise bibliográfica do que já foi escrito sobre o tema da sua pesquisa. Relacionar e distinguir o que já foi publicado (livros e revistas científicas) com o que está sendo proposto.

\section{Metodologia (materiais e métodos)}  
descrever como será realizado o trabalho, ou seja, quais os mecanismos utilizados e qual a forma da apresentação final.  Indicar como serão coletadas as informações (dados, entrevistas, etc.). Pela Metodologia ou procedimentos metodológicos, você deverá responder como? Onde? (Com que?) será desenvolvido o trabalho

\section{Cronograma}  
Identifica o período em que será realizada cada etapa da pesquisa. Pode ser apresentado em forma de tabela.




%incluir as referencias bibiográfica
\bibliographystyle{plain}
\bibliography{referencias}


\end{document}


